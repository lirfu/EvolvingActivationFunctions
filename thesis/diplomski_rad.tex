\documentclass[times, utf8, diplomski]{fer}
\usepackage{booktabs}

\begin{document}

\thesisnumber{1966}

\title{Optimizirane izlazne funkcije klasifikatora temeljenog na umjetnim neuronskim mrežama u domeni implementacijskih napada na kriptografske uređaje}

\author{Juraj Fulir}

\maketitle

% Ispis stranice s napomenom o umetanju izvornika rada. Uklonite naredbu \izvornik ako želite izbaciti tu stranicu.
\izvornik

\zahvala{ZAHVALA'n'STUFF}

\tableofcontents

\chapter{Uvod}
Opis problema 

\chapter{Implementacijski napadi na kriptografske uređaje}

\section{Side-channel napadi}
Postoji nekoliko vrsta.

Ovdje se obrađuje DPA.

\section{Izvedba napada}
Uštekaj uređaj, osciloskop na to i to mjesto i snimaj

Provjeri mogućnosti i zaključi najvjerojatniju

Problem netraktabilnosti postupka -> neuralke <3

\section{DPA skupovi podataka}
Ima HW i ovaj pravi

Nabaci i PCA redukcije i statistike iz jn

Mjere dobrote klasifikacije

Ne zaboravi referencu na stranicu!

\chapter{Klasifikator temeljen na umjetnim neuronskim mrežama}

\section{Umjetne neuronske mreže}
Svojstva kompresije i generalizacije

Problem odabira arhitekture, hiperparametara i optimizacije

\section{Izlazne funkcije}
Nomenklatura izlazne/prijenosne fje (ona 2 cool rada)

Bitka za odabir akrivacijske fje (nađi onaj rad di pljuje po sigmoidi i relu (elu rad?))

\chapter{Optimizacija simboličkom regresijom (tehnički genetskim programiranjem...)}

\section{Simbolička regresija}
Opis i svojstva SR

Utjecaj i brojnost parametara u GA (moš linkat i svoj završni rad :P)

\section{Taboo evolucijski algoritam}
Problem konvergencije i stohastičnosti GP-a

EA oplemenjen taboo listom iz algoritma Taboo pretraživanja

\section{Korišteni čvorovi i operatori}
Popis čvorova

Popis operatora (un/bin)

\chapter{Implementacija}
???

\chapter{Rezultati}

\section{9class}

\subsection{Uobičajene izlazne funkcije}
Opis postupka pretrage

Tablica

Komentar

\subsection{Utjecaj parametra veličine taboo liste}
Tablica

Komentar

\section{256class}

\subsection{Uobičajene izlazne funkcije}
Opis postupka pretrage

Tablica

Komentar

% Hoće bit vremena za ovo?
\subsection{Utjecaj parametra veličine taboo liste}
Tablica

Komentar

\chapter{Buduća istraživanja}
Primjena CNN na vremenskim uzorcima po uzoru na onaj rad

Ispitivanje učinkovitosti korištene optimizacije na ostalim problemima

Paralelna evolucija arhitekture i aktivacijskih fja

\chapter{Zaključak}
Radi/Ne radi. 

Pronađene zanimljivosti. 

Pouka za doma.

\bibliography{literatura}
\bibliographystyle{fer}

\begin{sazetak}
Proučiti postojeće metode u izgradnji izlaznih funkcija u umjetnim neuronskim mrežama. Posebnu pažnju posvetiti evolucijskim algoritmima simboličke regresije za izgradnju ciljanih funkcija. Ustanoviti moguće nedostatke postojećih algoritama ili mogućnost poboljšanja. Primijeniti evoluirane izlazne funkcije u homogenoj ili heterogenoj umjetnoj neuronskoj mreži na skupovima DPAv2 i DPAv4 te odrediti mjere kvalitete izgrađenog klasifikatora: točnost, preciznost, odziv te F mjere. Usporediti učinkovitost ostvarenih postupaka s postojećim rješenjima iz literature. Radu priložiti izvorne tekstove programa, dobivene rezultate uz potrebna objašnjenja i korištenu literaturu.

\kljucnerijeci{Ključne riječi, odvojene zarezima.}
\end{sazetak}

% TODO: Navedite naslov na engleskom jeziku.
\engtitle{Title}
\begin{abstract}
Abstract.

\keywords{Keywords.}
\end{abstract}

\end{document}
